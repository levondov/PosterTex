%% ****** Start of file auguide.tex ****** %
%%
%%   This file is part of the AIP distribution of substyles for REVTeX 4.1
%%   For version 4.1r of REVTeX, August 2010
%%
%%   Copyright (c) 2009,2010 American Institute of Physics
%%
\listfiles
\documentclass[%
 reprint,%
%secnumarabic,%
 amssymb, amsmath,%
 aip,cha,%
%groupedaddress,%
%frontmatterverbose,
]{revtex4-1}

\usepackage{docs}%
\usepackage{bm}%
\usepackage[colorlinks=true,linkcolor=blue]{hyperref}%
%\nofiles
\expandafter\ifx\csname package@font\endcsname\relax\else
 \expandafter\expandafter
 \expandafter\usepackage
 \expandafter\expandafter
 \expandafter{\csname package@font\endcsname}%
\fi
\hyphenation{title}

\begin{document}

\title{Streak Camera Systems and Tune Resonance Tools at HZB}%

\author{Levon Dovlatyan, Markus Ries, Paul Goslawski, and friends at HZB}%
\email{levondov@berkeley.edu}
\affiliation{University of California Berkeley,94720 Berkeley, CA, USA}%

\date{August 2014}%
%\revised{August 2010}%

\maketitle

\tableofcontents

\section{Introduction}

The Helmholtz-Zentrum Berlin is a facility that operates two synchrotron light sources: BESSY II and the MLS. One of the long term goals of this center is to continually make bunch lengths as short as possible in storage rings. A way to go about this is through lattice design by changing the momentum compaction factor. An important part of lattice design involves picking a good working point to avoid the tune resonances of the machine. A tune resonance program was therefore developed which can be used to view the current working point and resonance lines given only a few input parameters from the EPICS control systems.

Another way of looking into bunch lengths at HZB is through diagnostic tools. A new streak camera was recently purchased for the MLS, and is currently being setup and tested. The Metrology Light Source (MLS) is an electron storage ring designed as a dedicated UV and VUV source; it has an asymmetric double-bend achromate design as well as six beamlines, two of which are located on a second floor, above the synchrotron. Both the new and the old streak cameras are setup with a UV beamline on the second floor, directly above the machine. Being on the second floor and having the cameras setup far away from the beamline required advanced optical paths in order to get a focused beam into the slit of the cameras.

\section{Tune Resonance Program}

\subsection{Developement}
The program was written in Python. This provided the opportunity to integrate with EPICS (Experimental Physics and Industrial Control System) through a special python package called PyEpics and build a GUI system using wxPython to make the program simple to use. Much of the program also relies heavily on numpy and matplotlib packages available for Python.

The biggest feature of the program is a 'live mode' option that is able to give the current tune position numerically and graphically. Using the proper epics channels, the program is able to grab the four values it needs to calculate the non integer tune: the RF cavity frequency, the harmonic number, and the horizontal and vertical betatron oscillation frequencies. The program can check and update the tune several times a second, but in order to have the visual display updated, a threading process was used to avoid having the GUI freeze up while in live mode. The threading process is able to do all the processing and updating of the matplotlib graph, continously updating the point on the graph.



\subsection{Customizability}

\subsection{Future Improvements}

\section{\revtex\ Class Options Specific to AIP}

\subsection{Journal Substyle}
To access particular features of the AIP substyle, you will specify an additional document class option: the journal substyle, e.g.,
\begin{verbatim}
\documentclass[aip,jcp]{revtex4-1}
\end{verbatim}
in this case, \textit{J. Chem. Phys.}, the default. 
A complete list of AIP journals with the corresponding journal substyle appears in Table~\ref{tab:journals}.

\subsection{Options for Citations and Bibliography}
The citation style for AIP journals is:
\begin{itemize}
\item 
numerical (default style), 
\item
author-year, and
\item
numerical author-year,
\end{itemize}
the latter two styles being only allowed for \textit{Chaos} or \textit{J. Math. Phys.}

The familiar numerical citations and numbered bibliography are the default for most journals: 
citations are superscript numbers, and the (numbered) bibliographic entries appear in the order cited. 

Author-year citations are only allowed for 
\textit{Chaos} or \textit{J. Math. Phys.}, with citations given in author-and-year format. 
Bibliographic entries are sorted by alphabetical order of first author's surname, then by year. 

Numerical author-year citations 
(only allowed for \textit{Chaos} or \textit{J. Math. Phys.}) 
are superscript numbers, just like numerical citations, 
but the bibliographic entries are sorted like the author-year entries and are numbered. 
This means that the first citation will not necessarily be~1.

To obtain the numerical style, simply accept the default, or supply a class option of \texttt{numerical}:
\begin{verbatim}
\documentclass[aip,numerical]{revtex4-1}
\end{verbatim}
For author-year citations for \textit{Chaos} or \textit{J. Math. Phys.}, 
you may specify the \texttt{author-year} option:
\begin{verbatim}
\documentclass[aip,author-year]{revtex4-1}
\end{verbatim}
Each of the above two options are part of standard \revtex.

To obtain numerical author-year citations 
for \textit{Chaos} or \textit{J. Math. Phys.}, 
give the author-numerical option:
\begin{verbatim}
\documentclass[aip,author-numerical]{revtex4-1}
\end{verbatim}
Note that the \texttt{author-numerical} option is not part of standard \revtex\, so use of it
outside of the AIP substyles may not have any effect. 

\subsection{Formatting Options}
There are two commonly used formats for an article you may write. 
One will comply with the manuscript submission formatting requirements of the editorial office of the journal you are submitting to.
The other will emulate the format of your article in the published journal itself. 

For journal submission, accept the default, or you may specify the \texttt{preprint} option:
\begin{verbatim}
\documentclass[aip,preprint]{revtex4-1}
\end{verbatim}
To emulate the formatting of the journal, specify the \texttt{reprint} option:
\begin{verbatim}
\documentclass[aip,reprint]{revtex4-1}
\end{verbatim}
Note that emulation is not by any means complete: the fonts used will differ, and therefore
the length of the article will not represent an accurate estimate. 
Other details may also differ. 

A summary of class options of interest to AIP authors appears in Table~\ref{tab:options}.
\begin{table}
\caption{\label{tab:options}Other class options}
\begin{ruledtabular}
\begin{tabular}{ll}
\textbf{Function} & \textbf{class option} \\
\multicolumn{2}{l}{\textit{Citation and References}}\\
superscript numbered&\texttt{numerical}\footnotemark[1]\textsuperscript{,}\footnotemark[2]\\
author-year&\texttt{author-year}\footnotemark[3]\\
numbered author-year&\texttt{author-numerical}\footnotemark[3]\\
%
\multicolumn{2}{l}{\textit{Format}}\\
journal submission&\texttt{preprint}\footnotemark[1]\\
journal emulation&\texttt{reprint}\\
\end{tabular}
\end{ruledtabular}
\footnotetext[1]{Default option.}%
\footnotetext[2]{Standard}%
\footnotetext[3]{Only allowed for \textit{Chaos} or \textit{J. Math. Phys.}}%
\end{table}

\section{Useful \LaTeXe\ Markup}
\LaTeXe\ markup is the preferred way to structure your file. 
In general, the use of low-level commands like \TeX\ primitives or Plain \TeX\ macros 
is less preferable. 
Please see the \revtex\ User's Guide,\cite{Note2} 
the \LaTeX\ manual,\cite{LaTeXman} 
and the \LaTeXe\ book\cite{Compan} 
for further details. 

\subsection{Title and Front Matter}\label{sec:front}

The \revtex\ User's Guide has complete information on using \revtex's special markup for your
article's title, author list, abstract, and other front matter elements. 
Note that class option \texttt{superscriptaddress} is the default for the AIP substyles, 
as required by all AIP journals. 

\subsection{Lead Paragraph}
One AIP journal, \textit{Chaos}, requires a paragraph of text to precede the first
\cmd\section\ of the article; 
this is known as a lead paragraph and is formatted boldface. 
To give your article a lead paragraph, 
include a quotation environment ahead of the first \cmd\section\ command:
\begin{verbatim}
\documentclass[aip]{revtex4-1}
\begin{document}
 \begin{quotation}
  Here is my lead paragraph!
 \end{quotation}
 \section{Introduction}
...
\end{verbatim}

The \texttt{quotation} environment functions normally after the first \cmd\section\ command in the document.

\section{Body}

For general information on commands used in the body of the document, see the \revtex\ User's Guide.
Herein are some features specific to the AIP author.

\subsection{Footnotes}

If you are using numbered citations (numerical or numbered author-year), 
footnotes are by default incorporated into the reference section 
along with your bibliographic entries. 
This automated feature is only effective if you use Bib\TeX\ to prepare your
bibliography. 

Author-year style bibliography does not lend itself to such a treatment, 
so by default footnotes appear in text as is usual. 
However, be advised that, if your article is accepted for publication,
footnotes may be incorporated into text during the production process.

\section{Citations and References}\label{sec:endnotes}

The preparation of your bibliography ``by hand'' is possible; 
however, if you do so, 
you will be entirely responsible 
for compliance with submission requirements for your bibliographic entries, 
for incorporating any text footnotes into the references, 
and for checking bibliographic entries. 
(In this connection, you may find useful the file \texttt{reftest.tex}, distributed with \revtex.)

There are numerous reasons to use Bib\TeX, not least because it automates the first and second of the above checks. 

\subsection{\label{sec:use-bib}Using Bib\protect\TeX}

Refer to the \revtex\ User's Guide, the \LaTeX\ manual, and the Bib\TeX\ manual
for full information about using Bib\TeX. 

When using Bib\TeX\, keep in mind that changing your bibliography style or citation style
(via the document class options described above) will require you to rerun Bib\TeX.
The standard litany (using \texttt{aipsamp.tex} as an example) for this is:
\begin{verbatim}
> latex aipsamp
> bibtex aipsamp
> latex aipsamp
> latex aipsamp
\end{verbatim}
Here, the first invocation of \texttt{latex} has the effect of rewriting the
\texttt{aipsamp.aux} file,
and the invocation of \texttt{bibtex} creates a new \texttt{aipsamp.bbl} file. 
The next two runs of \texttt{latex} are then required: 
the first to update the \texttt{aipsamp.aux} file reflecting the new values of your citations
and the second to employ those citations correctly. 
Be sure to check the end of the \texttt{aipsamp.log} file for any message advising you to 
rerun \texttt{latex}. 

\subsection{Multiple References per Citation}
In an article using numerical citations, 
it is not uncommon to encounter the need for a citation 
that refers to more than one article or other reference. 
To accommodate such a case, \revtex~4.1 implements markup similar to that of the 
\texttt{mcite} package for \LaTeXe. 

Let's say that two citation keys \texttt{able} and \texttt{baker} 
need to be combined into a single reference.
The syntax for the \cmd\cite\ command is:
\begin{verbatim}
word\cite{able,*baker} further text
\end{verbatim}
When you run Bib\TeX\, the resulting bibliography will contain the two entries, but run together
as a single numbered reference.
In the \cmd\cite\ command argument, any cite key that starts with the * character
signifies that its bibliographic entry is to be joined together with the one preceding it;
the \texttt{*} may join together any number of entries into a single reference.

\begin{thebibliography}{9}\label{sec:TeXbooks}%
\bibitem{Note1}
For help regarding the installation of this software and its use, please send email to \href{mailto:tex@aip.org}{tex@aip.org}.
%
\bibitem{Note2}
Available with the \revtex\ distribution, see \url{http://authors.aps.org/revtex4/}.
%
\bibitem[Lamport(1996)]{LaTeXman} 
L. Lamport, 
\emph{\LaTeX\, a Document Preparation System} 
(Addison-Wesley, Reading, MA, 1996).
%
\bibitem[Goossens(1994)]{Compan} 
M. Goosens, F. Mittelbach, and A. Samarin, 
\emph{The \LaTeX\ Companion} 
(Addison-Wesley, Reading, MA, 1994).
%
\bibitem[Knuth(1986)]{TeXbook} 
D. E. Knuth, 
\emph{The \TeX book} 
(Addison-Wesley, Reading, MA, 1986). 
%
\bibitem[Kopka(1995)]{Guide} 
H. Kopka and P. Daly, 
\emph{A Guide to \LaTeXe} 
(Addison-Wesley, Reading, MA, 1995).
%
\bibitem[Goossens(1997)]{CompanG} 
M. Goossens, S. Rahtz, and F. Mittelbach, 
\emph{The \LaTeX\ Graphics Companion} 
(Addison-Wesley, Reading, MA, 1997).
%
\bibitem[Rahtz(1999)]{CompanW} 
S. Rahtz, M. Goossens \emph{et al.},
\emph{The \LaTeX\ Web Companion} 
(Addison-Wesley, Reading, MA, 1999).
%
\end{thebibliography}

\end{document}

